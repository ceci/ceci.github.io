\documentclass[a4paper, 10pt, onecolumn]{report}
\usepackage[left=1in, top=1in, right=1in, nohead, nofoot]{geometry}
\usepackage[dvips]{graphicx}
\usepackage[usenames,dvipsnames]{xcolor}
\usepackage{hyperref}

\hypersetup{colorlinks=false,
			urlcolor=black,
			urlbordercolor=white,
			pdfborderstyle={/S/U/W 0.5}
			}

\newcommand{\linkImage}{\includegraphics[height=2ex]{linkIcon}}

\pagestyle{empty}
% Name and contact information
\newcommand{\name}{Cecilia Fernandes}
\newcommand{\addr}{R. Prof. Maria Pacheco Nobre, 35}
\newcommand{\compl}{04326-110 Sao Paulo, SP - Brazil}
\newcommand{\github}{http://github.com/ceci}
\newcommand{\email}{contact@cecifernandes.com}
\newcommand{\phone}{+55-11-99133-8144}
\newcommand{\site}{http://cecifernandes.com}

\begin{document}

%The document itself
\begin{center}{\Huge\scshape\name}\end{center}
\vspace{-8pt} \rule{\textwidth}{1pt}
\vspace{-1pt} {\it{\small \addr \hfill \email \\\compl \hfill \site \\\phone \hfill \github}}
\\
\\
{\large\textbf{Experience}} \hspace{0.5em}\hrulefill

\begin{itemize}
\item{\textbf{Caelum (March/2008 - present)}\\
\indent{Caelum is a well known training company in Brazil, providing logics, Java, and Agile courses, among others. My work at Caelum involves innovation through software development and the study of new technologies and concepts, teaching, and mentoring at Caelum and at clients, specially in Agile techniques, processes, and team building.

What is unique about my work is that I naturally got in charge of creating, maintaining and thoughtfully changing the company's culture as Caelum grew from eighteen to more than a hundred people. Nowadays, I provide coaching to colleagues, make Agile interventions on the organization level, and lead the team that builds Caelum's tailor-made ERP $-$ this team is well known for succesfully transforming interns in instructors and leaders, at Caelum.
\\www.caelum.com.br}}

\item{\textbf{Agile Brazil (July/2011 - present)}\\
\indent{Agile Brazil is the largest Agile conference in the South Hemisphere and it is completely organized by passionate agilists in their own time. It has been a valuable experience because it is a year-round work that consistently delivers a one thousand attendees conference with a geographically distributed, ever changing group of organizers.

As a team player on the core comittee and chair to Program, I have created an environment that allowed everyone in my team to collaborate, criticize, and enhance not only their outcome but also the process that lead to it. I was also responsible for the creation of the Volunteers program, which comes with its own challenges: conference volunteers often meet for the first time the day before the event and they have to become a team almost immediately $-$ and that has been not only achieved, but also multiplied by those at their local conferences.
\\www.agilebrazil.com}}

\item{\textbf{IBM Research (December/2007 - March/2008)}\\
\indent{I was one of three chosen undergraduate interns for the K42 group in TJ Watson IBM Research. I have hacked and debugged an object oriented C++ filesystem in order to make it work with the 2.6 version of Linux kernel.
\\www.research.ibm.com/K42/}}

\item{\textbf{Caelum (July/2007 - December/2007)}\\
\indent{As an intern at Caelum, I developed mainly opensource software, including Caelum Stella, a library focused on the Brazilian developer needs, and Tubaina, a textbook generator used by both Caelum and a Brazilian technical books publisher named Casa do C\'{o}digo. I was also one of the people responsible for keeping the classrooms infrastructure running through the administration of 45 to 90 Linux running computers.}}
\end{itemize}
\\
{\large\textbf{Education}} \hspace{0.5em}\hrulefill
\\
\\\indent\textbf{2005 - 2012:} BSc in Computer Science - University of Sao Paulo - Brazil.
\\
\\
{\large\textbf{Languages}} \hspace{0.5em}\hrulefill
\begin{itemize}
	\item{\textbf{Portuguese:} fluent}
	\item{\textbf{English:\ \ \ \ \ } fluent}
	\item{\textbf{French:\ \ \ \ \ } elementary}
\end{itemize}
\\
\\
{\large\textbf{Skills}} \hfill \rule{5.7in}{0.5pt}\\

\\\textbf{Computer Languages:}
\begin{itemize}
	\item{Proficient in Java, Javascript, \textit{\LaTeX}, Bash, HTML5, CSS3}	%Expert
	\item{Have coded in Groovy, C, C++, Ruby, Scala}			%Proficient?
\end{itemize}
\\
\\
\noindent{\large\textbf{Personal initiatives}} \hspace{0.5em}\hrulefill
\begin{itemize} \itemsep1pt \parskip0pt \parsep0pt
	\item{\textbf{Speaker at conferences}\\
	\indent{I am a frequent speaker in conferences concerning Agile methods and software development due to my strong belief that knowledge is best when spread. Here are some of the conferences in which I was a speaker:

		\begin{itemize}
			\item{\textbf{Agile Brazil - 2010 to 2015}\\
			\indent{\textbf{2015:} Volunteers at Agile Brazil -- Building a performing team from scratch, year after year, on \textit{General Interest} track}}\\
			\\
			\indent{\textbf{2014:} Freedom at the Workplace\href{http://cecifernandes.com/liberdade_no_trabalho/}{\linkImage}, with Raphael Molesim, on \textit{Coaching \& Facilitation} track}}\\
			\indent{\textbf{2014:} Gender diversity in IT \href{http://cecifernandes.com/igualdade_de_genero/en/}{\linkImage}, with Mariana Bravo, on \textit{General Interest} track}}\\
			\\
			\indent{\textbf{2013:} Rewriting the Agile Principles, a workshop \href{http://www.slideshare.net/WilliamSeitiMizuta/agile-brazil-2013-23920449}{
			\linkImage}, with William Mizuta\\
			\\
			\indent{\textbf{2012:} Novice on your agile team? Beware of the inertia!\href{http://www.infoq.com/br/presentations/novatos-time-agil}{\linkImage}, on \textit{Management and Culture} track}}\\
			\\
			\indent{\textbf{2011:} Giants in Ivory Towers \href{http://www.slideshare.net/guilhermecaelum/o-grandiosismo-dos-loucos-agile-brazil-2011-cecilia-fernandes-e-guilherme-silveira}{\linkImage}, with Guilherme Silveira, on \textit{General Interest} track}}\\
			\\
			\indent{\textbf{2010:} When Scrum got in our way \href{http://www.infoq.com/br/presentations/quando-scrum-passou-a-atrapalhar}{\linkImage}, with Guilherme Silveira, on \textit{Experience Reports} track}}\\

			\item{\textbf{Agile in Rio - 2013}
			\href{http://www.slideshare.net/cecifernandes/broken-windows-de-prticas-geis}{\linkImage}\\
			\indent{Agile practices broken windows, on \textit{Engineering} track}}\\

			\item{\textbf{Agile at Dallas - 2012}
			\href{http://www.slideshare.net/cecifernandes/there-andbackagainagile2012
			}{\linkImage}\\
			\indent{There and Back Again: From Iterative to Flow... and Back to Iterative!, on \textit{Experience Reports} track}}\\


			\item{\textbf{QCon S\~{a}o Paulo - 2011}
			\href{http://www.slideshare.net/cecifernandes/melhorando-um-ambiente-gil/}{\linkImage}\\
			\indent{Improving an Already Agile Environment, on \textit{State of Art Agile} track}}\\
		\end{itemize}
	}}\\
	\item{\textbf{Volunteer at conferences}\\
	\indent{
	Volunteering is my favorite way of participating in conferences and meeting interesting and motivated people. I have been a volunteer at QCon San Francisco 2010 and 2013, XP Madrid 2011, QCon New York 2012, and many other Brazilian conferences and what I find particularly exciting about it is that volunteer programs often show that providing people autonomy can extract the best from them.
	}}\\
\end{itemize}
\\
\\
{\large\textbf{About me}} \hspace{0.5em}\hrulefill

\\Improvement drives me. I am a very curious person who is always willing to learn something new and acknowledges that every individual has something to add to my life. As a trained and seasoned instructor, I have developed strong didactics, which I also find very useful while dealing with teams.

\\Software development and Agile are both work and fun for me and mixing those two topics deeply energizes me.

\\Also, the passion for improvement is certainly applied to work, and not limited to it. For that reason, I have a quite wide personal background, from ballroom dancing to skydiving. Learning and trying new activities makes me happy, and gives me a broad range of subjects I am comfortable with. 
\end{document}
